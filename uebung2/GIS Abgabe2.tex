\documentclass{article}

\usepackage{fancyhdr} % Required for custom headers
\usepackage{lastpage} % Required to determine the last page for the footer
\usepackage{extramarks} % Required for headers and footers
\usepackage{graphicx} % Required to insert images
\usepackage{lipsum} % Used for inserting dummy 'Lorem ipsum' text into the template
\usepackage{amssymb}
\usepackage{colortbl}
\usepackage{amsmath}
\newcommand{\Mod}[1]{\ (\mathrm{mod}\ #1)}
\usepackage{mathtools}
\DeclarePairedDelimiter\ceil{\lceil}{\rceil}
\DeclarePairedDelimiter\floor{\lfloor}{\rfloor}


% Margins
\topmargin=-0.45in
\evensidemargin=0in
\oddsidemargin=0in
\textwidth=6.5in
\textheight=9.0in
\headsep=0.55in 

\linespread{1.1} % Line spacing

% Set up the header and footer
\pagestyle{fancy}
\lhead{\hmwkAuthorName} % Top left header
\rhead{\hmwkClass\ (\hmwkClassInstructor): \hmwkTitle} % Top right header
\chead{} % Top right header
\lfoot{\lastxmark} % Bottom left footer
\cfoot{} % Bottom center footer
\rfoot{Page\ \thepage\ of\ \pageref{LastPage}} % Bottom right footer
\renewcommand\headrulewidth{0.4pt} % Size of the header rule
\renewcommand\footrulewidth{0.4pt} % Size of the footer rule

\setlength\parindent{0pt} % Removes all indentation from paragraphs

%----------------------------------------------------------------------------------------
%	DOCUMENT STRUCTURE COMMANDS
%	Skip this unless you know what you're doing
%----------------------------------------------------------------------------------------

% Header and footer for when a page split occurs within a problem environment
\newcommand{\enterProblemHeader}[1]{
\nobreak\extramarks{#1}{#1 continued on next page\ldots}\nobreak
\nobreak\extramarks{#1 (continued)}{#1 continued on next page\ldots}\nobreak
}

% Header and footer for when a page split occurs between problem environments
\newcommand{\exitProblemHeader}[1]{
\nobreak\extramarks{#1 (continued)}{#1 continued on next page\ldots}\nobreak
\nobreak\extramarks{#1}{}\nobreak
}

\setcounter{secnumdepth}{0} % Removes default section numbers
\newcounter{homeworkProblemCounter} % Creates a counter to keep track of the number of problems

\newcommand{\homeworkProblemName}{}
\newenvironment{homeworkProblem}[1][Aufgabe \arabic{homeworkProblemCounter}]{ % Makes a new environment called homeworkProblem which takes 1 argument (custom name) but the default is "Problem #"
\stepcounter{homeworkProblemCounter} % Increase counter for number of problems
\renewcommand{\homeworkProblemName}{#1} % Assign \homeworkProblemName the name of the problem
\section{\homeworkProblemName} % Make a section in the document with the custom problem count
\enterProblemHeader{\homeworkProblemName} % Header and footer within the environment
}{
\exitProblemHeader{\homeworkProblemName} % Header and footer after the environment
}

\newcommand{\problemAnswer}[1]{ % Defines the problem answer command with the content as the only argument
\noindent\framebox[\columnwidth][c]{\begin{minipage}{0.98\columnwidth}#1\end{minipage}} % Makes the box around the problem answer and puts the content inside
}

\newcommand{\homeworkSectionName}{}
\newenvironment{homeworkSection}[1]{ % New environment for sections within homework problems, takes 1 argument - the name of the section
\renewcommand{\homeworkSectionName}{#1} % Assign \homeworkSectionName to the name of the section from the environment argument
\subsection{\homeworkSectionName} % Make a subsection with the custom name of the subsection
\enterProblemHeader{\homeworkProblemName\ [\homeworkSectionName]} % Header and footer within the environment
}{
\enterProblemHeader{\homeworkProblemName} % Header and footer after the environment
}
   
%----------------------------------------------------------------------------------------
%	NAME AND CLASS SECTION
%----------------------------------------------------------------------------------------

\newcommand{\hmwkTitle}{\"{U}bungsblatt \#2} % Assignment title
\newcommand{\hmwkDueDate}{Dienstag,\ November\ 21,\ 2017} % Due date
\newcommand{\hmwkClass}{Grundlagen der Informationssicherheit\ WS 2017/2018} % Course/class
\newcommand{\hmwkClassTime}{} % Class/lecture time
\newcommand{\hmwkClassInstructor}{Gruppenabgabe} % Teacher/lecturer
\newcommand{\hmwkAuthorName}{Marco Hildenbrand, \linebreak Felix B\"{u}hler, \linebreak Lukas Baur } % Your name

%----------------------------------------------------------------------------------------
%	TITLE PAGE
%----------------------------------------------------------------------------------------

\title{
\vspace{2in}
\textmd{\textbf{\hmwkClass:\ \hmwkTitle}}\\
\normalsize\vspace{0.1in}\small{Due\ on\ \hmwkDueDate}\\
\vspace{0.1in}\large{\textit{\hmwkClassInstructor\ \hmwkClassTime}}
\vspace{3in}
}

\author{\textbf{\hmwkAuthorName}}
\date{} % Insert date here if you want it to appear below your name

%----------------------------------------------------------------------------------------

\begin{document}

\maketitle

%----------------------------------------------------------------------------------------
%	TABLE OF CONTENTS
%----------------------------------------------------------------------------------------

%\setcounter{tocdepth}{1} % Uncomment this line if you don't want subsections listed in the ToC

%\newpage
%\tableofcontents
\newpage

%----------------------------------------------------------------------------------------
%	Aufgabe 1
%----------------------------------------------------------------------------------------
\begin{homeworkProblem}

\textbf{Getting the code $(a,b)$} \\ 
Als erstes sendet der Angreifer eine Anfrage zum verschl\"{u}sseln von 0 (also E(0,(a,b)). Zur\"{u}ckgegeben wird demnach:\
$E(x,(a,b)) = E(x,(a,b)) = (0*a) +_n b = b $. \\
Im Anschluss sendet der Angreifer eine Anfrage zum Verschl\"{u}sseln von 1. Es gilt offensichtlich:\\

$E(x,(a,b)) = E(1,(a,b)) = (1*a) +_n b = a +_n b$
Nun kann a leicht bestimmt werden, da E(1,(a,b)) sowie b bekannt ist. (a berechnet sich aus $E(1,(a,b)) - b$ bzw. $n + (E(1,(a,b)) - b$) falls  \textit{$E(1,(a,b)) - b$} negativ ist.)\\


\textbf{Win the game with advantage = 1 } \\ 
Da der Schl\"{u}ssel $(a,b)$ nun bekannt ist, kann trivialerweise jeder Cifer-Text damit entschl\"{u}sselt werden. \\
Dass der advantage demzufolge bei 1 liegt, d\"{u}rfte offensichtlich sein.\\Da der Angreifer den Code nun dechiffrieren kann, kann er das Tupel $(z_0 , z_1)$ senden und erh\"{a}lt $z_i =: c$.\\
Er verschl\"{u}sselt nun (ggf. eigenst\"{a}ndig) $z_0$ und $z_1$. Nun kann er vergleichen ob E($z_0$ , $(a,b)$) = $c$ gilt. Falls ja, so war $i=0$, sonst $i = 1.$
\\
Alternativ entschl\"{u}sselt er $c$ mit $(a,b)$ und erh\"{a}lt $z_0$ oder $z_1$ und entscheidet entsprechend.\\\\



\end{homeworkProblem}

%----------------------------------------------------------------------------------------
%	Aufgabe 2
%----------------------------------------------------------------------------------------
\begin{homeworkProblem}
\textbf{Aufgabe 2.1} \\
s
 \\\\\\
\end{homeworkProblem}

%----------------------------------------------------------------------------------------
%	Aufgabe 3
%----------------------------------------------------------------------------------------
\begin{homeworkProblem}
Wir schreiben um:  \\
$a = p_1 + r_1 $ mit $0 \leq r1 < n$ und $p_1 =  k * n, k \in \mathrm{N}$ und demnach offensichtlich $r_1 = a$ mod $n$
\\
$b = p_2 + r_2 $ mit $0 \leq r2 < n$ und $p_2 =  \tilde{k} * n, \tilde{k} \in \mathrm{N}$ und demnach offensichtlich $r_2 = b$ mod $n$\\
\\Dann gilt:
$(a * b)$ mod $n \\= (p_1 + r_1)* ( p_2 + r_2)$ mod $n \\= (p_1 p_2 + r_1 p_2 + p_1 r_2 + r_1 r_2 )$ mod $n \\=
p_1 p_2 $ mod $n  + r_1 p_2 $ mod $n  + p_1 r_2 $ mod $n  + r_1 r_2 $ mod $n  \\= $
$k n \tilde{k}n $ mod $n  + r_1 n\tilde{k} $ mod $n  + k n r_2 $ mod $n  + r_1 r_2 $ mod $n  \\= 
0  + 0 + 0 + r_1 r_2 $ mod $n  \\= $
$ r_1 r_2 $ mod $n$
$\\= (a $ mod $ n)(b $ mod $ n)$mod $n$ \textit{(nach Definition)}
\\$\square$


\end{homeworkProblem}

%----------------------------------------------------------------------------------------
%	Aufgabe 4
%----------------------------------------------------------------------------------------
\begin{homeworkProblem}
Da $(R, +, *)$ ein kommutativer Ring ist, gilt:
\begin{list}{•}{•}
\item $(R, +, *)$ ist assoziativ
\item in $(R, +, *)$ existiert ein neutrales Element e bzg. der Multiplikation
\item $(R, +, *)$ ist distributiv
\item $(R, +, *)$ ist kommutativ bez\"{u}glich *.
\end{list}

$R^{*}$ ist nun definiert als die Menge aller invertierbaren Elemente aus $(R, +, *)$. \\
\\
$\mathrm{Z\kern-.3em\raise-0.5ex\hbox{Z}}$: 
\begin{enumerate}
\item $(x*y)*z = x*(y*z) \forall x,y,z \in R^{*}$
\item $\exists x^{-1} \in R^{*}: x*x^{-1} = x^{-1}*x = e, \forall x\in R^{*}$
\item $\exists e \in R^{*}: x*e = e*x = x, \forall x\in R^{*}$.
\end{enumerate}

1. Da $R^{*} \subseteq R$ ist, gilt (1) offensichtlich immer noch.\\\\
2. Nach Voraussetzung besteht $R^{*}$ nur aus invertierbaren Elementen, also existiert auch ein Inverses $e$ in $R^{*}$:\\
Sei $x$ invertierbar in $R^{*}$, dann existiert ein  $x^{-1} \in R^{*}$, das dessen Inverse bildet (nach Definition von $R^{*}$). Da $e*e^{-1} = e^{-1}*e = 1 \Leftrightarrow e^{-1} = e$. Da $e \in R$ war, und offensichtlich invertierbar ist, so ist es auch $\in R^{*}$
\\\\
3. Da $(R, +, *)$ ist kommutativ bez\"{u}glich * war, und f\"{u}r jedes y $\in R $ ein neutrales Element existiert, so gilt auch $x*e = e*x = x, \forall x\in R^{*}$ sofern, dieses $e$ auch in $R^{*}$ vorhanden ist. Dies ist gem\"{a}\ss \ (2) erf\"{u}llt.

Also erf\"{u}llt $(R^{*}, *)$ alle Gruppenaxiome. $\square$
\end{homeworkProblem}

%----------------------------------------------------------------------------------------
%	Aufgabe 5
%----------------------------------------------------------------------------------------
\begin{homeworkProblem}

$ ggT(a, b) := d\\
ggT(b, a \mod b) := e $\\
$\mathrm{Z\kern-.3em\raise-0.5ex\hbox{Z}}$: d = e\\
Hilfssatz:\\
$ d | a \land d | b\\
q = \floor*{\frac{a}{b}}\\
a \mod b = a - qb$\\
also: $ d | (a-qb) $ Lemma Linearkombination\\
also $ d | (a \mod b) $\\
da $ d | a \mod b \land d | b $\\
$ \Rightarrow d | e $\\~\\
$ e = ggT(b, a \mod b) \Rightarrow e | b$ und $ e | a \mod b $\\
$ \Rightarrow e | (a \mod b - qb) $\\
$ \Rightarrow e | a $\\
$ \Rightarrow e | ggT(a, b) = d $\\
$ \Rightarrow e | d $\\
$ \Rightarrow e | d \land d | e \Leftrightarrow e = d $
\end{homeworkProblem}

%----------------------------------------------------------------------------------------
%	Aufgabe 6
%----------------------------------------------------------------------------------------
\begin{homeworkProblem}
ExtendedEuclid(32,51):\\
\begin{tabular}{c|c|c|c|c|c|c|c}
 
$a'$ & $b'$ & $x_0$ & $y_0$ & $x_1$ & $y_1$ & $q$ & $r$ \\ 
\hline 
32 & 51 & 1 & 0 &  \cellcolor{green}0 & \cellcolor{red}1 & 0 & \cellcolor{blue}32 \\ 
\hline 
51 & \cellcolor{blue}32 & \cellcolor{green}0 & \cellcolor{red}1 & \cellcolor{green}1 & \cellcolor{red}0 & 1 & \cellcolor{blue}19 \\  
\hline 
32 & \cellcolor{blue}19 & \cellcolor{green}1 & \cellcolor{red}0 & \cellcolor{green}-1 & \cellcolor{red}1 & 1 & \cellcolor{blue}13 \\  
\hline 
19 & \cellcolor{blue}13 & \cellcolor{green}-1 & \cellcolor{red}1 & \cellcolor{green}2 & \cellcolor{red}-1 & 1 & \cellcolor{blue}6 \\  
\hline 
13 & \cellcolor{blue}6 & \cellcolor{green}2 & \cellcolor{red}-1 & \cellcolor{green}-3 & \cellcolor{red}2 & 2 & \cellcolor{blue}1\\   
\hline 
6 & \cellcolor{blue}1 & \cellcolor{green}-3 & \cellcolor{red}2 & \cellcolor{green}8 & \cellcolor{red}-5 & 6 & \cellcolor{blue}0 \\   
\hline 
1 & \cellcolor{blue}0 & \cellcolor{green}8 & \cellcolor{red}-5 & -51 & 32 & • & • \\
\end{tabular} 
\\
\\$1 = 8*32 + (-5)*51$\\
$32^{-1} =_{51} 8$\\
$32^{-1} * 32 $ mod $51 = 8*32$ mod $ 51 = 256 $ mod $51 = 1$
$\square$ 

\end{homeworkProblem}
\end{document}

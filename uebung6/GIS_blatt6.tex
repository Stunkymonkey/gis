\documentclass[12pt,pdftex,a4paper]{article}
\usepackage[ngerman]{babel}
\usepackage[utf8]{inputenc}
\usepackage{amsmath}
\usepackage{amssymb}
\usepackage{ulem}
\usepackage{bbm}
\usepackage{array}
\usepackage{marvosym}
\usepackage{color}
\usepackage{hhline}
\usepackage[pdftex]{graphicx}
\usepackage{listings}
\lstset{language=Python,basicstyle=\footnotesize}
\usepackage{pdfpages}
\usepackage{booktabs}
\PassOptionsToPackage{hyphens}{url}
\usepackage{hyperref}
\usepackage{extarrows}
\usepackage{rotating}
\newcommand\tab[1][1cm]{\hspace*{#1}}

\title{ Grundlagen Informationsicherheit und Datensicherheit,\\ Blatt 6}
\author{Lukas Baur, 3131138\\
	Felix Bühler, 2973410\\
	Marco Hildenbrand, 3137242}

\begin{document}
\maketitle
\section*{Problem 1: Protocol for Mutual Authentication}
Reflection attack:\\
Eine Person, die $ N_B $ korrekt entschlüsselt hat, ist jemand, der den KEY kennt (Alice). Allerdings kennt Bob selbst den KEY auch! Der Angreifer kann also die gesendeten Nachrichten auzeichnen und später nochmal senden und sich damit als Alice ausgeben.
\\~\\
E = Evil (= Attacker)
\begin{enumerate}
	\item Verbindung 1: A $ \rightarrow $ B \tab : $ enc_s^k(N_A) $
	\item Verbindung 1: B $ \rightarrow $ A \tab : $ enc_s^k(N_B), N_A $
	\item Verbindung 1: A $ \rightarrow $ B \tab : $ N_B $
	\setlength{\itemsep}{20pt}
	\item Verbindung 2: E $ \rightarrow $ A \tab : $ enc_s^k(N_B) $
	\setlength{\itemsep}{5pt}
	\item Verbindung 2: A $ \rightarrow $ E \tab : $ enc_s^k(N_A'), N_B $
	\item Verbindung 3: E $ \rightarrow $ A \tab : $ enc_s^k(N_A') $
	\item Verbindung 3: A $ \rightarrow $ E \tab : $ enc_s^k(N_A''), N_A' $
	\item Verbindung 2: A $ \rightarrow $ E \tab : $ N_A' $
\end{enumerate}

\section*{Problem 2: Kerberos}


\section*{Problem 3: TLS and Randomness}
\begin{sidewaystable}[ph!]
	\begin{tabular}{|c|c|c|c|c|}
		C &  $\xlongrightarrow{\text{(1) Client hello:} TLSVersion_{C},  CipherSuites_{C}, N_{C}}$  & E &  $\xlongrightarrow{\text{(1) Client hello:} TLSVersion_{C},  CipherSuites_{C}, N_{C}}$   & S \\ 
		
		C &  $\xlongleftarrow{\text{(2) Server hello:} TLSVersion_{S},  ChosenCipherSuites_{S}, N_{S}, CERT_{E}(k_{E})}$ & E &  $\xlongleftarrow{\text{(2) Server hello:} TLSVersion_{S},  ChosenCipherSuites_{S}, N_{S}, CERT_{S}(k_{S})}$ & S \\ 
		
		C &  $\xlongrightarrow{\text{(3)Client Key Exchange: } enc^{a}_{k_{E}}(PMS)}$  & E &  $\xlongrightarrow{\text{(3)Client Key Exchange: } enc^{a}_{k_{S}}(PMS)}$  & S \\ 
		
		C &  $\xlongrightarrow{\text{(4)Client Finished: } enc_{k}^{auth}(hash(MS, TranscriptOfHandshake_{CE}))}$  & E &  $\xlongrightarrow{\text{(4)Client Finished: } enc_{k}^{auth}(hash(MS, TranscriptOfHandshake_{ES}))}$  & S \\ 
		
		C &  $\xlongleftarrow{\text{(5)Server Finished: } enc_{k}^{auth}(hash(MS, TranscriptOfHandshake_{EC}))}$ & E &  $\xlongleftarrow{\text{(5)Server Finished: } enc_{k}^{auth}(hash(MS, TranscriptOfHandshake_{SE}))}$ & S \\ 
		& • &  & • &  \\ 
		& • &  & • &  \\ 
		& • &  & • &  \\ 
		C & \textit{Session beendet} & E & \textit{Session beendet} & S \\ 
	\end{tabular} 
	\\\\
	\\
	\begin{tabular}{|c|c|c|c|c|}
		C &  $\xlongrightarrow{\text{(1) Client hello:} TLSVersion_{C},  CipherSuites_{C}, N_{C}}$  & E &  $\xlongrightarrow{\text{(1) Client hello:} TLSVersion_{C},  CipherSuites_{C}, N_{C}}$   & S \\ 
		
		C &  $\xlongleftarrow{\text{(2) Server hello:} TLSVersion_{S},  ChosenCipherSuites_{S}, N_{S}, CERT_{C}(k_{S})}$ & E &  $\xlongleftarrow{\text{(2) Server hello:} TLSVersion_{S},  ChosenCipherSuites_{S}, N_{S}, CERT_{C}(k_{S})}$ & S \\ 
		
		C &  $\xlongrightarrow{\text{(3)Client Key Exchange: } enc^{a}_{k_{S}}(PMS)}$  & E &  $\xlongrightarrow{\text{(3)Client Key Exchange: } enc^{a}_{k_{S}}(PMS')}$  & S \\ 
		
		C &  $\xlongrightarrow{\text{(4)Client Finished: } enc_{k}^{auth}(hash(MS, TranscriptOfHandshake))}$  & E &  $\xlongrightarrow{\text{(4)Client Finished: } enc_{k}^{auth}(hash(MS, TranscriptOfHandshake))}$  & S \\ 
		
		C &  $\xlongrightarrow{\text{(5)Server Finished: } enc_{k}^{auth}(hash(MS, TranscriptOfHandshake'))}$ & E &  $\xlongrightarrow{\text{(5)Server Finished: } enc_{k}^{auth}(hash(MS, TranscriptOfHandshake'))}$ & S \\ 
		C & • & E & • & S \\ 
		C & • & E & • & S \\ 
		C & • & E & • & S \\ 
		C & \textit{Session beendet} & E & \textit{Session beendet} & S \\ 
	\end{tabular} 
\end{sidewaystable}
\clearpage

\section*{Problem 4: Attack on IKEv1 Main Mode}
Reflection attack:

Der Angriff ist eine einfache Reflexion, die erfordert, dass ein Initiator seine eigene Identität als Gleichrangigen-Peer akzeptiert. In diesem Fall werden die Authentifizierer in dem Protokoll gleich und der Gegner sendet einfach alle Nachrichten, die von dem Initiator kommen, unverändert an sie zurück. In diesem Fall schlägt die schwache Vereinbarung fehl, da kein Agent die Responder-Rolle ausführt. Die Geheimhaltung des Schlüssels ist weiterhin gewährleistet. Wenn eine Selbstkommunikation nicht möglich ist, ist der aufgezeigte Angriff nicht mehr möglich.

\end{document}

\documentclass{article}

\usepackage{fancyhdr} % Required for custom headers
\usepackage{lastpage} % Required to determine the last page for the footer
\usepackage{extramarks} % Required for headers and footers
\usepackage{graphicx} % Required to insert images
\usepackage{lipsum} % Used for inserting dummy 'Lorem ipsum' text into the template
\usepackage{amssymb}
\usepackage{colortbl}
\usepackage[utf8]{inputenc}


% Margins
\topmargin=-0.45in
\evensidemargin=0in
\oddsidemargin=0in
\textwidth=6.5in
\textheight=9.0in
\headsep=0.55in 

\linespread{1.1} % Line spacing

% Set up the header and footer
\pagestyle{fancy}
\lhead{\hmwkAuthorName} % Top left header
\rhead{\hmwkClass\ (\hmwkClassInstructor): \hmwkTitle} % Top right header
\chead{} % Top right header
\lfoot{\lastxmark} % Bottom left footer
\cfoot{} % Bottom center footer
\rfoot{Page\ \thepage\ of\ \pageref{LastPage}} % Bottom right footer
\renewcommand\headrulewidth{0.4pt} % Size of the header rule
\renewcommand\footrulewidth{0.4pt} % Size of the footer rule

\setlength\parindent{0pt} % Removes all indentation from paragraphs

%----------------------------------------------------------------------------------------
%	DOCUMENT STRUCTURE COMMANDS
%	Skip this unless you know what you're doing
%----------------------------------------------------------------------------------------

% Header and footer for when a page split occurs within a problem environment
\newcommand{\enterProblemHeader}[1]{
\nobreak\extramarks{#1}{#1 continued on next page\ldots}\nobreak
\nobreak\extramarks{#1 (continued)}{#1 continued on next page\ldots}\nobreak
}

% Header and footer for when a page split occurs between problem environments
\newcommand{\exitProblemHeader}[1]{
\nobreak\extramarks{#1 (continued)}{#1 continued on next page\ldots}\nobreak
\nobreak\extramarks{#1}{}\nobreak
}

\setcounter{secnumdepth}{0} % Removes default section numbers
\newcounter{homeworkProblemCounter} % Creates a counter to keep track of the number of problems

\newcommand{\homeworkProblemName}{}
\newenvironment{homeworkProblem}[1][Aufgabe \arabic{homeworkProblemCounter}]{ % Makes a new environment called homeworkProblem which takes 1 argument (custom name) but the default is "Problem #"
\stepcounter{homeworkProblemCounter} % Increase counter for number of problems
\renewcommand{\homeworkProblemName}{#1} % Assign \homeworkProblemName the name of the problem
\section{\homeworkProblemName} % Make a section in the document with the custom problem count
\enterProblemHeader{\homeworkProblemName} % Header and footer within the environment
}{
\exitProblemHeader{\homeworkProblemName} % Header and footer after the environment
}

\newcommand{\problemAnswer}[1]{ % Defines the problem answer command with the content as the only argument
\noindent\framebox[\columnwidth][c]{\begin{minipage}{0.98\columnwidth}#1\end{minipage}} % Makes the box around the problem answer and puts the content inside
}

\newcommand{\homeworkSectionName}{}
\newenvironment{homeworkSection}[1]{ % New environment for sections within homework problems, takes 1 argument - the name of the section
\renewcommand{\homeworkSectionName}{#1} % Assign \homeworkSectionName to the name of the section from the environment argument
\subsection{\homeworkSectionName} % Make a subsection with the custom name of the subsection
\enterProblemHeader{\homeworkProblemName\ [\homeworkSectionName]} % Header and footer within the environment
}{
\enterProblemHeader{\homeworkProblemName} % Header and footer after the environment
}
   
%----------------------------------------------------------------------------------------
%	NAME AND CLASS SECTION
%----------------------------------------------------------------------------------------

\newcommand{\hmwkTitle}{\"{U}bungsblatt \#3} % Assignment title
\newcommand{\hmwkDueDate}{Dienstag,\ December\ 05,\ 2017} % Due date
\newcommand{\hmwkClass}{Grundlagen der Informationssicherheit\ WS 2017/2018} % Course/class
\newcommand{\hmwkClassTime}{} % Class/lecture time
\newcommand{\hmwkClassInstructor}{Gruppenabgabe} % Teacher/lecturer
\newcommand{\hmwkAuthorName}{Lukas Baur, \linebreak Felix B\"{u}hler, \linebreak Marco Hildenbrand} % Your name

%----------------------------------------------------------------------------------------
%	TITLE PAGE
%----------------------------------------------------------------------------------------

\title{
\vspace{2in}
\textmd{\textbf{\hmwkClass:\ \hmwkTitle}}\\
\normalsize\vspace{0.1in}\small{Due\ on\ \hmwkDueDate}\\
\vspace{0.1in}\large{\textit{\hmwkClassInstructor\ \hmwkClassTime}}
\vspace{3in}
}

\author{\textbf{\hmwkAuthorName}}
\date{} % Insert date here if you want it to appear below your name

%----------------------------------------------------------------------------------------

\begin{document}

\maketitle

%----------------------------------------------------------------------------------------
%	TABLE OF CONTENTS
%----------------------------------------------------------------------------------------

%\setcounter{tocdepth}{1} % Uncomment this line if you don't want subsections listed in the ToC

%\newpage
%\tableofcontents
\newpage

%----------------------------------------------------------------------------------------
%	Aufgabe 1
%----------------------------------------------------------------------------------------
\begin{homeworkProblem}
 
\begin{table}[!ht]
\centering
\caption{Fast-Exponentiation}
\label{my-label}
\begin{tabular}{lll}
i  & h        & k        \\
3  & 1        & 9        \\
2  & 9        & 81 $\equiv$ 4   \\
1  & 36       & 16       \\
0  & 36       & 256 $\equiv$ 25 \\
-1 & 900 $\equiv$ 53 & 9       
\end{tabular}
\end{table}

Das Ergebnis ist demnach  53.\\



\end{homeworkProblem}

%----------------------------------------------------------------------------------------
%	Aufgabe 2
%----------------------------------------------------------------------------------------
\begin{homeworkProblem}
\textbf{Problem 2a} \\
Nein, es ist keine kollisions-resistente Hash-Funktion: Da jeder Eingabewerte denselben Hash-Wert ergibt, sobald die Anzahl der 1en identisch sind, ist es trivial eine zweite Nachricht erzeugen, die dieselbe Anzahl an Einsen hat. Falls x kein Pallindrom ist, reicht es zum Beispiel aus, die Eingabe zu spiegeln, andernfalls reich das Permutieren der Eingabe, damit ein $\tilde{x}$ ensteht, sodass $H(x) = H(\tilde{x})$ gilt.\\

\textbf{Problem 2b} \\
Nein, es ist keine kollisions-resistente Hash-Funktion, dazu berechnen wir zu einem x ein $\tilde{x}$, sodass $h(x) = h(\tilde{x})$ gilt: \\
Sei $x_0$ beliebig aber fest der Länge L-1. Dessen Hash-Wert ist dann $h(x_0)$ (mit der Länge l). Nun erstellen wir $x = h(x_0) || h(x_0)$. $x$ ist der Länge $> L$ aber $<2L$ und $H(x)$ wird dadurch gebildet mit $H(x) = h(x_0) \oplus h(x_0) = a \oplus a = 0^l$\\
Bestimme nun ein $x_1$ beliebig aber fest der Länge L-1. Dessen Hash-Wert ist dann $h(x_1)$ (mit der Länge l). Nun erstellen wir $\tilde{x} = h(x_1) || h(x_1)$. Analog zu dem ersten Teil haben wir für $H(\tilde{x})$ wieder $0^l$. Damit wurde eine Kollision gefunden.$\square$
 \\\\\\
\end{homeworkProblem}

%----------------------------------------------------------------------------------------
%	Aufgabe 3
%----------------------------------------------------------------------------------------
\begin{homeworkProblem}
Im Folgenden zeigen wir, dass folgendes gilt: \\
$h$ kollisionsresistent $=>$ $h'$ kollisionsresistent \\\\

Wir können aus einer Kollision für $h'$ eine Kollision für h berechnen.  Ergebe $x$ und $x'$ eine Kollision für $h'$: $h'(x) = x(0)|h(x) = x'(0)|h(x') = h'(x')$. daraus folgt ganz offensichtlich dass $h(x) = h(x')$ gilt. Also ist $(x, x')$ auch eine Kollision unter h. \\\\

\textit{Sei $h$ kollisionsresistent} \\
Dann existiert kein effizientes Verfahren, mit dem zu einem $x$ leicht ein $x'$ berechnet werden kann, sodass $h(x) = h(x')$ gilt. Wir nehmen im folgenden an, dass kein solches $x'$ eine Kollision existiert, besser gesagt, dass der Angreifer keines findet. Das Finden einer Kollision in h' kann sehr leicht auf das Finden einer Kollision in h reduziert werden. Will man nun zu x eine Kollision mit x' in h' finden, muss folgendes gelten: $h'(x) = x(0) || h(x) = x'(0) || h(x') = h'(x')$.
Damit ist das erste Bit für x' bereits bestimmt, es bleibt folgenden Problem zu lösen: h(x) = h(x'). Dieses Problem ist (hier) nicht lösbar, da gemäß der Annahme kein x' dazu berechnet werden kann.
$\square$
\end{homeworkProblem}

%----------------------------------------------------------------------------------------
%	Aufgabe 4
%----------------------------------------------------------------------------------------
\begin{homeworkProblem}
Das Security-Spiel Besteht aus 2 Phasen: \\\\
\textit{Phase 1: Anfragen}\\
Erhalte Mac für $x = x_0 = 0^l$ also $m_0 = E(x_0 \oplus v,k) = E(v,k)$\\
Erhalte Mac für $x' = x_0 || m_0$ also $m_1 = E(m_0 \oplus m_0,k) = E(0^l,k)$
\\
Erhalte Mac für $x'' = k$, k beliebiger Bit-String der Länger l, aber unterschiedlich zu $0^l$. Der erhaltene Mac heißt $m_2$
\\\\
\textit{Phase 2: Challenge} \\
Sende nun als Challenge ($x'' || m_2$, $m_1$). Da $x''$ den verschlüsselten Wert $m_2$ hat, gibt später $E(x'' \oplus v, k) \oplus m_2 = 0^l$ gilt $x'' || m_2$ ist selber Mac wie $m1$ \\

Damit kann ein Angreifer verschiedene Eingaben $x_i$ erstellen, die alle denselben Mac haben und davor noch nie an Alice zum mac-en gesendet worden sind. Die Wahrscheinlichkeit das Spiel zu gewinnen ist somit 1, also ist der Vorteil = 1. $\square$
\end{homeworkProblem}

%----------------------------------------------------------------------------------------
%	Aufgabe 5
%----------------------------------------------------------------------------------------
\begin{homeworkProblem}
Sei die darunterliegenden Hash-Funktion nicht kollisionsresistent und es kann demnach leicht ein x' zu einem x gefunden worden, sodass h(x) = h(x') gilt.\\
Jetzt gilt:\\
PKCS-sig$(x, (n,d)) = $PKCS-hash$(x)^d$ $mod$ n und außerdem \\
PKCS-sig$(x', (n,d)) = $PKCS-hash$(x')^d$ $mod$ n.\\
Da allerding PKCS-hash$(x)$ $mod$ n = PKCS-hash$(x')$ $mod$ n gilt, gilt auch \\ PKCS-hash$(x)^d$ $mod$ n = PKCS-hash$(x')^d$ $mod$ n sowie \\ PKCS-sig$(x, (n,d))$ = PKCS-sig$(x', (n,d))$.
Also ist es leicht möglich, eine gleiche Signatur zu erzeugen, wenn man eine nicht-kollisionsresistente Funktion darunterliegend implementiert.\\\\
\textit{Sicherheitsspiel mit adavantage 1:}
Lasse ein x von Alice signieren. Erhalte s.
Sende eine Challenge an Alice: $(x', s)$. Der Angreifer gewinnt immer, da der Hash derselbe ist, da x' hier so gewählt ist, dass es eine Kollision mit x gibt, und demnach auch der exponentierte Hash derstelbe ist. Die Wahrscheinlichkeit ist demnach 100\% und der Advantage demzufolge 1.

\end{homeworkProblem}

 
\end{document}

